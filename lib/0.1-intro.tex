\setcounter{page}{1}

\addcontentsline{toc}{chapter}{INTRODUCCIÓN}
\chapter*{INTRODUCCIÓN}
\markboth{INTRODUCCIÓN}{}
\label{chap0:introducion}
\justify %%justificar los párrafos

\hspace*{1em} El continuo avance tecnológico en el ámbito de las comunicaciones ópticas ha posibilitado la transmisión eficiente de vastas cantidades de información a través de redes de fibra óptica. Aunque estas redes presentan numerosas ventajas, su despliegue no está exento de desafíos, especialmente derivados de fenómenos no lineales \cite{Toulouse}. La interacción entre la luz y el material de la fibra óptica da lugar a efectos no lineales, destacando su relevancia en sistemas de \acrfull{wdm}, donde múltiples canales de diferentes longitudes de onda coexisten con mínima separación. La alta intensidad óptica, consecuencia de la propagación simultánea de canales de alta potencia, puede desencadenar fenómenos críticos que impactan la eficacia de la comunicación óptica. Entre estos fenómenos no lineales, destaca la \acrfull{fwm}, un efecto óptico no lineal de tercer orden. La \acrshort{fwm} surge cuando dos o más señales ópticas con distintas frecuencias centrales se propagan en una misma fibra, generando armónicos que presentan frecuencias equivalentes a la suma o diferencia de las ondas originales  \cite{Sing}\cite{Bansal}. 


En la búsqueda de soluciones para mitigar estos efectos no lineales, se han propuesto diversos enfoques. Para contrarrestar específicamente el impacto de la \acrshort{fwm}, se torna esencial una asignación eficiente de canales en las fibras ópticas; siendo una de las formas de mitigar el efecto \acrshort{fwm}, para definir los escenarios de prueba, se puede aumentar el espaciado de canales, menos canales, mejor distribución, aumento o disminución de la potencia, control de la dispersión cromática, entre otros. En este contexto, las \textit{reglas de Golomb}, originadas en la teoría de números y caracterizadas por definir conjuntos de enteros con la particularidad de tener todas sus sumas o diferencias distintas, emergen como una herramienta valiosa. Al garantizar distancias únicas entre las frecuencias asignadas, se puede mitigar el efecto de la \acrshort{fwm} y mejorar la calidad de la transmisión.


Sin embargo, el empleo de \textit{reglas de Golomb} para la asignación de canales no es una tarea trivial. El proceso de búsqueda y definición de estas reglas se vuelve computacionalmente desafiante, especialmente al considerar conjuntos de mayor tamaño.
Ante este panorama, surge la necesidad imperante de proponer un mecanismo dinámico que, aprovechando las propiedades matemáticas de las \textit{reglas de Golomb}, permita contrarrestar eficientemente el efecto de la \acrshort{fwm} en las \acrfull{MLR-PON}, adaptándose a diversas condiciones y requisitos de la red. En el contexto de canales \acrshort{wdm} uniformemente espaciados, la generación de términos \acrshort{fwm} en frecuencias adyacentes provoca diafonía entre canales, afectando el rendimiento general del sistema. La supresión de la \acrshort{fwm}, mediante la aplicación de \textit{reglas de Golomb}, puede tener implicaciones prácticas significativas, como la posibilidad de utilizar canales más densamente espaciados en el espectro óptico, aumentando así la capacidad del sistema. Además, la atenuación de la \acrshort{fwm} puede contribuir a mejorar la calidad de la señal, reducir la tasa de errores de bit y aumentar la distancia efectiva de transmisión \cite{Bansal}.

Con lo anterior, en el presente trabajo de investigación se plantea, a nivel de simulación, un mecanismo dinámico con enfoque en la asignación de canales con espaciado desigual, basado en \textit{reglas de Golomb}, en una arquitectura de red \acrshort{MLR-PON}. Ésto, con el próposito de analizar comparativamente la inclusión del algoritmo para constrarrestar los efectos de la \acrshort{fwm}, como sin él. A continuación, se describe el contenido definido en cuatro capítulos, en base a la información obtenida en el desarrollo de la investigación.



\noindent{\textbf{Capítulo 1: Generalidades sobre los Sistemas Ópticos, Fenómenos de Propagación, Velocidad de Transmisión de Línea Mixta y Reglas Golomb.}}

En este capítulo se describen algunos aspectos generales sobre los sistemas de telecomunicaciones basados en fibra óptica, describiendo a los sistemas WDM. Se realiza una caracterización de PON y luego se define el concepto de redes MLR y sus aspectos relevantes para el presente trabajo, como lo son los efectos líneales y no líneales que afectan los sistemas de comunicación ópticos, con especial énfasis en el fenómeno FWM. Finalmente, se realiza una descripción general de las reglas de Golomb. 

\noindent{\textbf{Capítulo 2: Desempeño Óptico, Marco Metodológico y Escenario de Simulación.}}

En este capítulo se describen los parámetros de desempeño óptico esensiales para el diseño y análisis de redes de comunicaciones ópticas.  También, se definen las metodologías y herramientas de simulación, mediante las cuales, se hará el diseño y desarrollo del modelo de red y el mecanismo dinámico, facilitando el planteamiento escenarios de simulación para el presente trabajo.


\noindent{\textbf{Capítulo 3: Análisis del Desempeño de Red MLR-PON Implementando un Mecanismo Dinámico en la Asignación de Canales.}}

En este capítulo se desarrolla el proceso de simulación que permitirá el análisis del desempeño de la red de MLR-PON frente a la FWM. Implementando un mecanismo dinámico de asignación de canales desigualmente espaciados, basado en reglas Golomb; como también, el análisis del desempeño de la red sin presencia del mecanismo.


\noindent{\textbf{Capítulo 4: Conclusiones, Recomendaciones y Trabajos Futuros.}}

En este capítulo se presentan conclusiones, recomendaciones y trabajos futuros relacionados con los aspectos más importantes de la presente investigación que permitieron diseñar e implementar un mecanismo dinámico de asignación de canales desigualmente espaciados en redes MLR-PON.
\clearpage

%\noindent{\textbf{Palabras clave:}}

%Mezcla de Cuadro Ondas (FWM), Multiplexación por División de Longitud de Onda (WDM), Red Óptica Pasiva (PON), Monitoreo de Desempeño Óptico (OPM), Reglas Golomb.



