%%------------- ACROYNMS ---------------
%--An acronym is a type of abbreviation formed from the initial letters of a phrase and pronounced as a word.
%--For example NASA (National Aeronautics and Space Administration)
%--all acronyms are abbreviations, but not all abbreviations are acronyms.

%------------------RED----------------------
\newacronym{pon}{PON}{Redes Ópticas Pasivas (\textit{Passive Optical network})}
\newacronym{wdm}{WDM}{Multiplexación por División de Longitud de Onda (\textit{Wavelength Division Multiplexing})}
\newacronym{USCA}{USCA}{Supresión de Cruce de Señales Acústicas}
\newacronym{MLR-PON}{MLR-PON}{Redes de Multiplexación por Longitud de Onda Pasiva Reconfigurable (\textit{Multi-Level Resilience Passive Optical Network})}
\newacronym{OTxNs}{OTxNs}{Sistemas Ópticos de Transmisión por Longitud de Onda y por Número de Serie (\textit{Wavelength Division Multiplexing Systems with Serial Number})}
\newacronym{UDWDM}{UDWDM}{Multiplexación por División de Longitud de Onda Ultra-Densa (\textit{Ultra-Dense Wavelength Division Multiplexing})}
\newacronym{mlr}{MLR}{Velocidad de Línea Mixta (\textit{Mixed-Line Rate})}
\newacronym{cwdm}{CWDM}{Multiplexación por División de Longitud de Onda Ampliamente Espaciada (\textit{Coarse Wavelength Division Multiplexing})}
\newacronym{dwdm}{DWDM}{Multiplexación por Longitud de Onda Densa (\textit{Dense Wavelength Division Multiplexing})}
\newacronym{isp}{ISP}{Proveedor de Servicios de Internet (\textit{Internet Service Provider})}
\newacronym{fttx}{FTTx}{Fibra hasta la X (\textit{Fiber to the X})}
\newacronym{ftth}{FTTH}{Fibra hasta el Hogar (\textit{Fiber to the Home})}
\newacronym{olt}{OLT}{Terminal de Línea Óptica (\textit{Optical Line Terminal})}
\newacronym{onu}{ONU}{Unidad de Red Óptica (\textit{Optical Network Unit})}
\newacronym{co}{CO}{Oficina Central (\textit{Central Office})}
\newacronym{man}{MAN}{Red de Área Metropolitana (\textit{Metropolitan Area Network})}
\newacronym{wan}{WAN}{Red de Área Amplia (\textit{Wide Area Network})}
\newacronym{rn}{RN}{Nodo Remoto (\textit{Remote Node})}
\newacronym{odn}{ODN}{Red de Distribución Óptica (Optical Distribution Network)}
\newacronym{otn}{OTN}{Redes de Transporte Óptico (\textit{Optical Transport Network})}
\newacronym{oxc}{OXC}{Conmutador Óptico (\textit{Optical Cross-Connect})}
\newacronym{EDFA}{EDFA}{Amplificadores de Fibra Dopada con Erbio (\textit{Erbium-Doped Fiber Amplifier})}
\newacronym{opm}{OPM}{Monitoreo de Desempeño Óptico (\textit{Optical Performance Monitoring})}
\newacronym{cd}{CD}{Dispersión Cromática (\textit{Chromatic Dispersion})}
\newacronym{pmd}{PMD}{Dispersión por Modo de Polarización (\textit{Polarization Mode Dispersion})}
\newacronym{isi}{ISI}{Interferencia entre Símbolos (\textit{ Intersymbol Interference})}
\newacronym{dcf}{DCF}{Fibras de Compensación de Dispersión (\textit{Dispersion Compensating Fiber})}
\newacronym{dsf}{DSF}{Fibras Conmutadas por Dispersión (\textit{Dispersion-Shifted Fiber})}
\newacronym{dgd}{DGD}{Retardo de Grupo Diferencial (\textit{Differential Group Delay})}
\newacronym{fwhm}{FWHM}{Ruido de Fase del Láser(\textit{Full Width at Half Maximum})}


%----efectos no lineales de tipo elástico----
\newacronym{fwm}{FWM}{Mezcla de Cuatro Ondas (\textit{Four Wave Mixing})}
\newacronym{xpm}{XPM}{Modulación de Fase Cruzada (\textit{Cross-Phase Modulation})}
\newacronym{spm}{SPM}{Automodulación de Fase (\textit{Self-Phase Modulation})}

\newacronym{sbs}{SBS}{Dispersión de Brillouin Estimulada (\textit{Stimulated Brillouin Scattering})}
\newacronym{srs}{SRS}{Dispersión de Raman Estimulada (\textit{Stimulated Raman Scattering})}

\newacronym{NLSE}{NLSE}{Ecuación de Schrödinger No Lineal (\textit{Nonlinear Schrödinger Equation})}
\newacronym{ASE}{ASE}{Emisión Espontánea Amplificada (\textit{Amplified Spontaneous Emission})}


%------------OTROS-------------------------
\newacronym{ogr}{OGR}{Regla de Golomb Óptima (\textit{Optical Golomb Ruler})}
\newacronym{gcd}{GCD}{Greatest Common Divisor}
\newacronym{lcm}{LCM}{Least Common Multiple}
\newacronym{pmi}{PMI}{Project Management Institute}
\newacronym{pmbok}{PMBOK}{Project Management Body of Knowledge}
\newacronym{tic}{TIC}{Tecnologías de la Información y de las Comunicaciones}
\newacronym{OOC}{OOC}{Códigos Ortogonales Ópticos (\textit{Optical Orthogonal Codes})}
\newacronym{ICI}{ICI}{Interferencia entre Portadoras (\textit{Inter-Carrier Interference})}
\newacronym{mzm}{MZM}{Modulador Mach-Zehnder(\textit{Mach-Zehnder Modulator})}
\newacronym{cw}{CW}{Onda Continua (\textit{Continuous Wave})}
\newacronym{dfb}{DFB}{Retroalimentación Distribuida (\textit{Distributed Feedback})}
\newacronym{vcsel}{VCSEL}{Láser de Emisión Vertical de Cavidad Superficial (\textit{Vertical-Cavity Surface-Emitting Laser})}
\newacronym{ccm}{CCM}{Componente Personalizado para MATLAB (\textit{Custom Component for MATLAB})}
\newacronym{vbs}{VBS}{Simulación con Ancho de Banda Variable (\textit{Variable Bandwidth Simulation Technique})}
\newacronym{spt}{SPT}{Técnica de Propagación Espectral (\textit{Spectral Propagation Technique})}


%------------MODULACIONES------------------
 \newacronym{RZ}{RZ}{Modulación de Retorno a Cero (\textit{Return-to-Zero})}
\newacronym{RZ-OOK}{RZ-OOK}{Modulación por Encendido y Apagado con Codificación Retorno a Cero (\textit{Return-to-Zero On-Off Keying})}
\newacronym{NRZ}{NRZ}{Modulación de No Retorno a Cero (\textit{Non-Return-to-Zero})}
\newacronym{NRZ-OOK}{NRZ-OOK}{Modulación por Encendido y Apagado con Codificación No Retorno a Cero (\textit{Non-Return-to-Zero On-Off Keying})}
\newacronym{RZ-DPSK}{RZ-DPSK}{Modulación por Desplazamiento de Fase Diferencial con Retorno a Cero (\textit{Return-to-Zero Differential Phase Shift Keying})}
\newacronym{QAM}{QAM}{Modulación de Amplitud en Cuadratura de 16 Estados (\textit{Quadrature Amplitude Modulation})}
\newacronym{PDM-QPSK}{PDM-QPSK}{Multiplexación por División de Polarización-Modulación de Desplazamiento de Fase en Cuadratura (\textit{Polarization Division Multiplexing-Quadrature Phase Shift Keying})}
\newacronym{DQPSK}{DQPSK}{Modulación por Desplazamiento de Fase en Cuadratura Diferencial (\textit{Differential Quadrature Phase Shift Keying})}
\newacronym{QPSK}{QPSK}{Modulación de Fase en Cuadratura (\textit{Quadrature Phase Shift Keying})}
\newacronym{CDMA}{CDMA}{Acceso Múltiple por División de Código (\textit{Code Division Multiple Access})}
\newacronym{NLFDM}{NLFDM}{Multiplexación por División de Frecuencia No Lineal (\textit{Non-Linear Frequency Division Multiplexing})}
\newacronym{OOK}{OOK}{Modulación por Encendido y Apagado (\textit{On-Off Keying})}
\newacronym{dpsk}{DPSK}{Modulación por Desplazamiento de Fase Diferencial (\textit{ Differential Phase Shift Keying})}



%------------PARÁMETROS OPM----------------
\newacronym{osnr}{OSNR}{Relación Señal-Ruido Óptica (\textit{Optical Signal-to-Noise Ratio})}
\newacronym{snr}{SNR}{Relación Señal-Ruido (\textit{Signal-to-Noise Ratio})}
\newacronym{ber}{BER}{Tasa de Error de Bits (\textit{Bit Error Rate})}















%%---- ABBREVIATIONS ----
%--An abbreviation is a shortened form of a word or phrase.
%--For example, the word abbreviation can itself be represented by the abbreviation abbr., abbrv., or abbrev.
\newglossaryentry{latex}
{
    name=latex,
    description={Is a markup language specially suited
            for scientific documents}
}
\newglossaryentry{maths}
{
    name=mathematics,
    description={Mathematics is what mathematicians do}
}


%%---- HOW TO USE ----
%--\gls{gcd} -> GCD
%--\gls{latex} -> latex - minúscula

%--\Gls{gcd} -> Greatest Common Divisor (GCD) - same as \acrfull{gcd}
%--\Gls{latex} -> Latex - mayúscula

%--\glspl{gcd} -> Greatest Common Divisors (GCDs)
%--\glspl{latex} -> Latexes

%--\Glspl{gcd} -> Greatest Common Divisors (GCDs)
%--\Glspl{latex} -> Latexes

%--\acrshort{gcd} -> GCD
%--\acrlong{gcd} -> Greatest Common Divisor
%--\acrfull{gcd} -> Greatest Common Divisor (GCD)
